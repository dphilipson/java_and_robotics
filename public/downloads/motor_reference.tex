\documentclass[11pt]{article}
\usepackage[noBBpl]{mathpazo}

%\documentclass[12pt]{article}
%\usepackage{mathptmx}

\usepackage{graphicx}
\usepackage{enumerate}
\usepackage{microtype}
\usepackage{amsfonts}
\usepackage{amsmath}
\usepackage{amssymb}
\usepackage{amsthm}
\usepackage{cancel}
\usepackage{mathrsfs}
\input xy
\xyoption{all}

\usepackage[noend]{algorithmic}
\usepackage{listings}
\lstset{language=C, tabsize=4, frame=single}

%\renewcommand{\labelenumi}{\textbf{(\arabic{enumi})}}

\pdfpagewidth 8.5in
\pdfpageheight 11in
\topmargin 0in
\headheight 0in
\headsep 0in
\textheight 9in
\oddsidemargin 0in
\evensidemargin 0in
\textwidth 6.5in

\theoremstyle{plain}
\newtheorem*{theorem}{Theorem}
\newtheorem*{lemma}{Lemma}
\newtheorem*{prop}{Proposition}
\newtheorem*{cor}{Corollary}

\theoremstyle{definition}
\newtheorem*{defn}{Definition}
\newtheorem*{prob}{Problem}
\newtheorem*{ex}{Example}
\newtheorem*{exes}{Examples}

\theoremstyle{remark}
\newtheorem*{remark}{Remark}
\newtheorem*{note}{Note}
\newtheorem*{claim}{Claim}
\newtheorem*{case}{Case}
\newtheorem*{conclusion}{Conclusion}

\newcommand{\N}{\mathbb N}
\newcommand{\Z}{\mathbb Z}
\newcommand{\Q}{\mathbb Q}
\newcommand{\R}{\mathbb R}
\newcommand{\C}{\mathbb C}

\newcommand{\paren}[1]{{\left({#1}\right)}}
\newcommand{\abs}[1]{{\left\lvert{#1}\right\rvert}}
\newcommand{\norm}[1]{{\left\lVert{#1}\right\rVert}}
\newcommand{\inner}[1]{{\left\langle{#1}\right\rangle}}
\newcommand{\floor}[1]{{\left\lfloor{#1}\right\rfloor}}
\newcommand{\ceil}[1]{{\left\lceil{#1}\right\rceil}}
\newcommand{\pd}[2]{{\frac{\partial{#1}}{\partial{#2}}}}
\newcommand{\unit}[1]{\,\mathrm{#1}}
\newcommand{\e}[1]{\times10^{#1}}
\newcommand{\bra}[1]{\langle #1 \rvert}
\newcommand{\ket}[1]{\lvert #1 \rangle}
\newcommand{\braket}[2]{\langle #1 | #2 \rangle}
\newcommand{\braaket}[3]{\bra{#1}#2\ket{#3}}

\DeclareMathOperator{\Real}{Re}
\DeclareMathOperator{\Imag}{Im}

\newcommand{\newproblem}[1]{\section*{\textsf{#1}\smallskip\hrule}}

\begin{document}
\begin{center}
\section*{LeJOS basic commands reference}
\end{center}
For printing on the LCD screen:
\begin{center}
  \begin{tabular}{|l|p{20em}|}
    \hline
    {\tt LCD.drawString(String str, int x, int y)} & Displays the provided
    string on the LCD screen, starting at position $(x, y)$. \\\hline
    {\tt LCD.drawInt(int n, int x, int y)} & Displays the integer {\tt n} on the
    LCD screen, starting at position $(x, y)$. \\\hline
    {\tt LCD.clear()} & Clears the LCD screen, removing whatever has been
    printed. \\\hline
  \end{tabular}
\end{center}
Methos for working with the buttons on the brick. Note that {\tt
Button.waitForPress()} is useful as the last command of a program, as otherwise
the program will terminate immediately and clear anything which was displayed on
the LCD.
\begin{center}
  \begin{tabular}{|l|p{20em}|}
    \hline
    {\tt Button.waitForPress()} & Waits for any button on the brick to be
    pressed before continuing with the program. Returns an {\tt int} which
    represents which button was pressed (ask for details). \\\hline
    {\tt Button.readButtons()} & Returns an {\tt int} which is $0$ if no buttons
    are currently pressed down, and greater than $0$ if any buttons are
    currently pressed down. It is possible to use this value to determine which
    buttons in particular are currently pressed, ask for details. \\\hline
  \end{tabular}
\end{center}
You can access the three motors via the constants {\tt Motor.A}, {\tt Motor.B},
and {\tt Motor.C}, which are instances of the class {\tt NXTRegulatedMotor}.
You can tell a motor to do things using the following methods, so for example to
tell motor A to start running forward, you would write {\tt
Motor.A.forward()}.
\begin{center}
  \begin{tabular}{|l|p{20em}|}
    \hline
    {\tt forward()} & Start the motor rotating forward. \\\hline
    {\tt backward()} & Start the motor rotation backward. \\\hline
    {\tt stop()} & Quickly stop the motor rotating. \\\hline
    {\tt getTachoCount()} & Returns the motor angle in degrees. \\\hline
    {\tt setSpeed(int speed)} & Sets the speed in degrees per second, up to a
    max of around 1000. \\\hline
    {\tt rotate(int angle)} & Rotates the motor the specified angle in degrees.
    \\\hline
    {\tt rotateTo(int angle)} & Rotates the motor until it's tachometer count
    (as returned by {\tt getTachoCount()}) equals the specified angle in degree.
    \\\hline
    {\tt rotate(int angle, true)} & The same as the other {\tt rotate()}, but
    returns immediately, i.e.\ the next command in the program runs immediately
    rather than waiting for the rotation to complete. \\\hline
    {\tt rotateTo(int angle, true)} & The same as the other {\tt rotateTo}, but
    returns immediately. \\\hline
    {\tt isMoving()} & Returns {\tt true} if the motor is currently moving.
    \\\hline
  \end{tabular}
\end{center}
There are some other, more obscure motor methods. These probably won't be as
useful, but are listed here for reference.
\begin{center}
  \begin{tabular}{|l|p{20em}|}
    \hline
    {\tt resetTachoCount()} & Resets the tachometer count, as returned by {\tt
    getTachoCount}, to zero. \\\hline
    {\tt setAcceleration(int accel)} & Sets the rate at which the motor changes
    speed, in degrees per second per second. Try using a small value like 200 to
    have your robot smoothly accelerate or decelerate. \\\hline
    {\tt getSpeed()} & Returns the speed in degrees per second, as set by {\tt
    setSpeed()}. \\\hline
    {\tt getActualSpeed()} & Returns the speed that the motor is actually
    currently moving, in degrees per second. \\\hline
    {\tt getAcceleration()} & Returns the acceleration in degrees per second per
    second, as set by {\tt setAcceleration()}. \\\hline
    {\tt getLimitAngle()} & Gets the angle to which the motor is currently
    rotating, after a call to {\tt rotate()} or {\tt rotateTo()}. \\\hline
    {\tt isRotating()} & Returns {\tt true} if the motor is currently moving due
    to a {\tt rotate()} or {\tt rotateTo()} command. Compare to {\tt
    isMoving()}, which returns {\tt true} if the motor is moving for any reason.
    \\\hline
    {\tt flt()} & Stop the motor gradually. Like {\tt stop()}, but allows the
    motor to stop on its own rather than using power to stop it. \\\hline
  \end{tabular}
\end{center}

\end{document}
