\documentclass[11pt]{article}
\usepackage[noBBpl]{mathpazo}

%\documentclass[12pt]{article}
%\usepackage{mathptmx}

\usepackage{graphicx}
\usepackage{enumerate}
\usepackage{microtype}
\usepackage{amsfonts}
\usepackage{amsmath}
\usepackage{amssymb}
\usepackage{amsthm}
\usepackage{cancel}
\usepackage{mathrsfs}
\input xy
\xyoption{all}

\usepackage[noend]{algorithmic}
\usepackage{listings}
\lstset{language=Java, tabsize=4}

%\renewcommand{\labelenumi}{\textbf{(\arabic{enumi})}}

\pdfpagewidth 8.5in
\pdfpageheight 11in
\topmargin 0in
\headheight 0in
\headsep 0in
\textheight 9in
\oddsidemargin 0in
\evensidemargin 0in
\textwidth 6.5in

\theoremstyle{plain}
\newtheorem*{theorem}{Theorem}
\newtheorem*{lemma}{Lemma}
\newtheorem*{prop}{Proposition}
\newtheorem*{cor}{Corollary}

\theoremstyle{definition}
\newtheorem*{defn}{Definition}
\newtheorem*{prob}{Problem}
\newtheorem*{ex}{Example}
\newtheorem*{exes}{Examples}

\theoremstyle{remark}
\newtheorem*{remark}{Remark}
\newtheorem*{note}{Note}
\newtheorem*{claim}{Claim}
\newtheorem*{case}{Case}
\newtheorem*{conclusion}{Conclusion}

\newcommand{\N}{\mathbb N}
\newcommand{\Z}{\mathbb Z}
\newcommand{\Q}{\mathbb Q}
\newcommand{\R}{\mathbb R}
\newcommand{\C}{\mathbb C}

\newcommand{\paren}[1]{{\left({#1}\right)}}
\newcommand{\abs}[1]{{\left\lvert{#1}\right\rvert}}
\newcommand{\norm}[1]{{\left\lVert{#1}\right\rVert}}
\newcommand{\inner}[1]{{\left\langle{#1}\right\rangle}}
\newcommand{\floor}[1]{{\left\lfloor{#1}\right\rfloor}}
\newcommand{\ceil}[1]{{\left\lceil{#1}\right\rceil}}
\newcommand{\pd}[2]{{\frac{\partial{#1}}{\partial{#2}}}}
\newcommand{\unit}[1]{\,\mathrm{#1}}
\newcommand{\e}[1]{\times10^{#1}}
\newcommand{\bra}[1]{\langle #1 \rvert}
\newcommand{\ket}[1]{\lvert #1 \rangle}
\newcommand{\braket}[2]{\langle #1 | #2 \rangle}
\newcommand{\braaket}[3]{\bra{#1}#2\ket{#3}}

\DeclareMathOperator{\Real}{Re}
\DeclareMathOperator{\Imag}{Im}

\newcommand{\newproblem}[1]{\section*{\textsf{#1}\smallskip\hrule}}

\begin{document}
\begin{center}
\section*{How to create a standalone program}
\end{center}
\section*{Introduction}
Eclipse can produce standalone Java programs called JAR files (for Java
ARchive). As long as your computer has a Java Runtine Environment (JRE)
installed, you should be able to run a .jar as a program by double-clicking it.
Normally, this is very easy to do with a few clicks, but because we're using the
non-standard ACM libraries and the Stanford plugin in the ConsolePrograms and
GraphicsPrograms we have been writing throughout the course, we have to go
through a fairly long process instead. Here are the steps required to produce a
standalone program.
\section{Adding a {\tt main()} method}
You may have wondered why we needed a {\tt main()} method in the LeJOS programs
but not in the ConsolePrograms and GraphicsPrograms. This is because the
Stanford Eclipse plugin is automatically adding a {\tt main()} method. Before we
can export to a JAR, we need to add this method back in. That is, inside your
main class (the one that extends ConsoleProgram or GraphicsProgram), you shoudld
add the code:
\begin{lstlisting}
public static void main(String[] args) {
   new MyClass().start(args);
}
\end{lstlisting}
In this code snippet, you should replace ``MyClass'' with whatever the name of
your class is. So for instance if at the top of your file you have the line
\begin{lstlisting}
public class RandomCircles extends GraphicsProgram {
\end{lstlisting}
then you should write RandomCircles in place of MyClass in the above code. 
\section{Using Eclipse's Export Wizard}
The next series of steps will be to use Eclipse's Export Wizard feature to
produce a JAR and a manifest file. We will later need to modify the manifest
file to specify that the ACM libraries should be included, but first perform the
following steps:
\begin{enumerate}
  \item Select the project you want to export in the left-hand pane (the
    ``Package Explorer''). At the top of the screen, go to the {\bf File} menu
    and select {\bf Export}. This should make an export window appear in the
    middle of the screen.
  \item On the ``choose export destination'' screen, select the {\bf Java}
    folder, and then select the option {\bf JAR file}. Click the {\bf Next >}
    button.
  \item On the ``JAR File Specification'' screen, check the box next to the name
    of your project in the left-hand box, and then select the location you want
    the JAR file saved to in the box labeled ``Select the export destination.''
    Then click {\bf Next >}.
  \item The next screen is ``JAR Packaging Options.'' This screen is not
    important for us, and you can simply click {\bf Next >} without changing
    anything.
  \item The next screen is ``JAR Manifest Specification,'' and requires several
    steps.
    \begin{itemize}
      \item First, near the bottom of the window, select the main class using
        the {\bf Browse} button.  The main class is the one to which we earlier
        added the {\tt main()} method in the first step above.
      \item Then click the radio button labelled ``Generate the manifest file.''
      \item Then check the box for ``Save the manifest in the workspace.''
      \item Then click {\bf Browse} next to the box for ``Manifest file,''
        navigate to the folder where your project exists, and enter ``manifest''
        as the file name. If you do this correctly, the box will likely be
        filled with something like ``/ProjectName/manifest''.
      \item Finally, click {\bf Next >}.
    \end{itemize}
\end{itemize}
After all of these steps, you should see the manifest file show up in the
package explorer under the current project.
\section{Editing the manifest file}
We need to edit the manifest file that was just created to tell it to use the
ACM libraries.
\end{document}
