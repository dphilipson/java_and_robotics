\documentclass[11pt]{article}
\usepackage[noBBpl]{mathpazo}

%\documentclass[12pt]{article}
%\usepackage{mathptmx}

\usepackage{graphicx}
\usepackage{enumerate}
\usepackage{microtype}
\usepackage{amsfonts}
\usepackage{amsmath}
\usepackage{amssymb}
\usepackage{amsthm}
\usepackage{cancel}
\usepackage{mathrsfs}
\input xy
\xyoption{all}

\usepackage[noend]{algorithmic}
\usepackage{listings}
\lstset{language=C, tabsize=4, frame=single}

%\renewcommand{\labelenumi}{\textbf{(\arabic{enumi})}}

\pdfpagewidth 8.5in
\pdfpageheight 11in
\topmargin 0in
\headheight 0in
\headsep 0in
\textheight 9in
\oddsidemargin 0in
\evensidemargin 0in
\textwidth 6.5in

\theoremstyle{plain}
\newtheorem*{theorem}{Theorem}
\newtheorem*{lemma}{Lemma}
\newtheorem*{prop}{Proposition}
\newtheorem*{cor}{Corollary}

\theoremstyle{definition}
\newtheorem*{defn}{Definition}
\newtheorem*{prob}{Problem}
\newtheorem*{ex}{Example}
\newtheorem*{exes}{Examples}

\theoremstyle{remark}
\newtheorem*{remark}{Remark}
\newtheorem*{note}{Note}
\newtheorem*{claim}{Claim}
\newtheorem*{case}{Case}
\newtheorem*{conclusion}{Conclusion}

\newcommand{\N}{\mathbb N}
\newcommand{\Z}{\mathbb Z}
\newcommand{\Q}{\mathbb Q}
\newcommand{\R}{\mathbb R}
\newcommand{\C}{\mathbb C}

\newcommand{\paren}[1]{{\left({#1}\right)}}
\newcommand{\abs}[1]{{\left\lvert{#1}\right\rvert}}
\newcommand{\norm}[1]{{\left\lVert{#1}\right\rVert}}
\newcommand{\inner}[1]{{\left\langle{#1}\right\rangle}}
\newcommand{\floor}[1]{{\left\lfloor{#1}\right\rfloor}}
\newcommand{\ceil}[1]{{\left\lceil{#1}\right\rceil}}
\newcommand{\pd}[2]{{\frac{\partial{#1}}{\partial{#2}}}}
\newcommand{\unit}[1]{\,\mathrm{#1}}
\newcommand{\e}[1]{\times10^{#1}}
\newcommand{\bra}[1]{\langle #1 \rvert}
\newcommand{\ket}[1]{\lvert #1 \rangle}
\newcommand{\braket}[2]{\langle #1 | #2 \rangle}
\newcommand{\braaket}[3]{\bra{#1}#2\ket{#3}}

\DeclareMathOperator{\Real}{Re}
\DeclareMathOperator{\Imag}{Im}

\newcommand{\newproblem}[1]{\section*{\textsf{#1}\smallskip\hrule}}

\begin{document}
\begin{center}
\section*{Assignment 7 -- Differential steering}
\end{center}
\subsection*{Problem 1}
If you haven't already, build a robot using two different motors to power the
wheels, one motor powering a wheel or wheels on the left-hand side, and the
other motor powering a wheel or wheels on the right-hand side.
\subsection*{Problem 2}
Without using the DifferentialPilot class, experiment with differential
steering. Write a program to make the robot follow a circular path. Don't worry
if the actual path isn't exactly circular or the robot doesn't quite make it
back to its start position.

According to the formulas we derived in class and the angular velocities at
which your program has the two wheels spin, what is the theoretical radius that
your robot should trace? Is this close to what you observe in real life? Write
your answers in your program as comments.
\subsection*{Problem 3}
Use the DifferentialPilot class to write a program which has the robot follow a
path which traces out the sides of a square. To practice decomposition, your
program should have a method {\tt traceSquare()} which takes a single double as
its argument representing the side length of the square which is to be traced.
\subsection*{Challenge problem}
Experiment with the DifferentialPilot {\tt arc()} method for different intput
values. How closely do the different values correspond to the real-life turning
ratio for your robot? How does your robot fare with particularly large or small
values for the turning ratio? If you don't remember the argument(s) to {\tt
arc()}, look it up in the LeJOS documentation.
\end{document}
