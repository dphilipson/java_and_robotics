\documentclass[11pt]{article}
\usepackage[noBBpl]{mathpazo}

%\documentclass[12pt]{article}
%\usepackage{mathptmx}

\usepackage{enumerate}
\usepackage{microtype}
\usepackage{amsfonts}
\usepackage{amsmath}
\usepackage{amssymb}
\usepackage{amsthm}
\usepackage{cancel}
\usepackage{mathrsfs}
\input xy
\xyoption{all}

\usepackage[noend]{algorithmic}
\usepackage{listings}
\lstset{language=C, tabsize=4, frame=single}

%\renewcommand{\labelenumi}{\textbf{(\arabic{enumi})}}

\pdfpagewidth 8.5in
\pdfpageheight 11in
\topmargin 0in
\headheight 0in
\headsep 0in
\textheight 9in
\oddsidemargin 0in
\evensidemargin 0in
\textwidth 6.5in

\theoremstyle{plain}
\newtheorem*{theorem}{Theorem}
\newtheorem*{lemma}{Lemma}
\newtheorem*{prop}{Proposition}
\newtheorem*{cor}{Corollary}

\theoremstyle{definition}
\newtheorem*{defn}{Definition}
\newtheorem*{prob}{Problem}
\newtheorem*{ex}{Example}
\newtheorem*{exes}{Examples}

\theoremstyle{remark}
\newtheorem*{remark}{Remark}
\newtheorem*{note}{Note}
\newtheorem*{claim}{Claim}
\newtheorem*{case}{Case}
\newtheorem*{conclusion}{Conclusion}

\newcommand{\N}{\mathbb N}
\newcommand{\Z}{\mathbb Z}
\newcommand{\Q}{\mathbb Q}
\newcommand{\R}{\mathbb R}
\newcommand{\C}{\mathbb C}

\newcommand{\paren}[1]{{\left({#1}\right)}}
\newcommand{\abs}[1]{{\left\lvert{#1}\right\rvert}}
\newcommand{\norm}[1]{{\left\lVert{#1}\right\rVert}}
\newcommand{\inner}[1]{{\left\langle{#1}\right\rangle}}
\newcommand{\floor}[1]{{\left\lfloor{#1}\right\rfloor}}
\newcommand{\ceil}[1]{{\left\lceil{#1}\right\rceil}}
\newcommand{\pd}[2]{{\frac{\partial{#1}}{\partial{#2}}}}
\newcommand{\unit}[1]{\,\mathrm{#1}}
\newcommand{\e}[1]{\times10^{#1}}
\newcommand{\bra}[1]{\langle #1 \rvert}
\newcommand{\ket}[1]{\lvert #1 \rangle}
\newcommand{\braket}[2]{\langle #1 | #2 \rangle}
\newcommand{\braaket}[3]{\bra{#1}#2\ket{#3}}

\DeclareMathOperator{\Real}{Re}
\DeclareMathOperator{\Imag}{Im}

\newcommand{\newproblem}[1]{\section*{\textsf{#1}\smallskip\hrule}}

\begin{document}
\begin{center}
\section*{Assignment 1 -- Challenge problems}
\end{center}
\subsection*{Challenge problem 1 -- MultiplierKarel}
Write a program which uses Karel to multiply. The starting position has two
piles of beepers next to each other. The goal is to produce a pile of beepers
whose size is the product of the original two piles. The specifications are as
follows:
\begin{itemize}
  \item Karel begins in the lower-left corner, facing east. There are two piles
    of beepers, one in the lower-left corner (the same as Karel's starting
    position), and one on the space immediately to the east, directly in front
    of Karel. The provided world {\tt Calculator.w} is an example of such a
    world.
  \item The objective is to produce a pile of beepers on any square other than
    the initial two which contains a number of beepers equal to the product of
    the sizes of the initial two piles.
  \item At termination, the sizes of the initial two piles should be the same as
    at the start. It is okay for Karel to change the sizes of these piles while
    the program runs, but Karel must restore these piles to their original sizes
    before the program terminates.
  \item Also at termination, there should be no beepers present in the world
    other than those in the three piles: the two initial piles, and a third pile
    representing the product.
  \item You may assume that the size of the world is at least $3\times 3$, and
    that there are no beepers initially present in the world other than the ones
    in the two starting piles.
\end{itemize}
\subsection*{Challenge Problem 2 -- ExplorerKarel}
Write a program to fully explore a world. Starting anywhere in a world of
unknown shape, Karel should place a beeper at every space in the world. The
specifications are as follows:
\begin{itemize}
  \item The world may be any shape or size, and Karel may start at any
    position. There may be walls in any location.
  \item At termination, there should be a beeper at every location which is
    accessible from Karel's starting point.
  \item You may assume that the world does not initially contain any beepers.
  \item You can try using the provided world {\tt Maze.w} to test your program.
\end{itemize}
\end{document}
