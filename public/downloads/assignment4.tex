\documentclass[11pt]{article}
\usepackage[noBBpl]{mathpazo}

%\documentclass[12pt]{article}
%\usepackage{mathptmx}

\usepackage{graphicx}
\usepackage{enumerate}
\usepackage{microtype}
\usepackage{amsfonts}
\usepackage{amsmath}
\usepackage{amssymb}
\usepackage{amsthm}
\usepackage{cancel}
\usepackage{mathrsfs}
\input xy
\xyoption{all}

\usepackage[noend]{algorithmic}
\usepackage{listings}
\lstset{language=C, tabsize=4, frame=single}

%\renewcommand{\labelenumi}{\textbf{(\arabic{enumi})}}

\pdfpagewidth 8.5in
\pdfpageheight 11in
\topmargin 0in
\headheight 0in
\headsep 0in
\textheight 9in
\oddsidemargin 0in
\evensidemargin 0in
\textwidth 6.5in

\theoremstyle{plain}
\newtheorem*{theorem}{Theorem}
\newtheorem*{lemma}{Lemma}
\newtheorem*{prop}{Proposition}
\newtheorem*{cor}{Corollary}

\theoremstyle{definition}
\newtheorem*{defn}{Definition}
\newtheorem*{prob}{Problem}
\newtheorem*{ex}{Example}
\newtheorem*{exes}{Examples}

\theoremstyle{remark}
\newtheorem*{remark}{Remark}
\newtheorem*{note}{Note}
\newtheorem*{claim}{Claim}
\newtheorem*{case}{Case}
\newtheorem*{conclusion}{Conclusion}

\newcommand{\N}{\mathbb N}
\newcommand{\Z}{\mathbb Z}
\newcommand{\Q}{\mathbb Q}
\newcommand{\R}{\mathbb R}
\newcommand{\C}{\mathbb C}

\newcommand{\paren}[1]{{\left({#1}\right)}}
\newcommand{\abs}[1]{{\left\lvert{#1}\right\rvert}}
\newcommand{\norm}[1]{{\left\lVert{#1}\right\rVert}}
\newcommand{\inner}[1]{{\left\langle{#1}\right\rangle}}
\newcommand{\floor}[1]{{\left\lfloor{#1}\right\rfloor}}
\newcommand{\ceil}[1]{{\left\lceil{#1}\right\rceil}}
\newcommand{\pd}[2]{{\frac{\partial{#1}}{\partial{#2}}}}
\newcommand{\unit}[1]{\,\mathrm{#1}}
\newcommand{\e}[1]{\times10^{#1}}
\newcommand{\bra}[1]{\langle #1 \rvert}
\newcommand{\ket}[1]{\lvert #1 \rangle}
\newcommand{\braket}[2]{\langle #1 | #2 \rangle}
\newcommand{\braaket}[3]{\bra{#1}#2\ket{#3}}

\DeclareMathOperator{\Real}{Re}
\DeclareMathOperator{\Imag}{Im}

\newcommand{\newproblem}[1]{\section*{\textsf{#1}\smallskip\hrule}}

\begin{document}
\begin{center}
\section*{Assignment 4 -- Interactive graphics programs}
\end{center}
This program modifies and builds upon the RandomCircles program from the
previous assignment. If you did not fully complete the RandomCircles assignment,
then it is recommended that you work on that first, at least until you develop
it enough to be able to implement the following program.

Modify your RandomCircles program to make a game in which the player attempts
to click on all the circles as fast as possible. The game works as follows:
\begin{itemize}
  \item First, a dialog box appears asking the player how many circles should be
    used in the game.
  \item After a number is chosen, that many random circles appear in the window,
    following the same rules as in the previous assignment. That is, their radii
    should be random between $5$ and $50$, their color should be random, and
    their position should be random, but chosen so that the circle fits entirely
    in the window.
  \item The player must click all the circles. Whenever the player clicks a
    circle, that circle disappears.
  \item Once no more circles are left on the screen, the player is given a
    message telling them that they have completed the game. This message can be
    a dialog window, or it can be a GLabel somewhere in the window.
\end{itemize}
After you have the basic game working, the next step is to add more features to
make it more fun. From here, what you do with the game is up to you. Here are
some ideas for additions, but you should feel free to come up with and implement
your own ideas:
\begin{itemize}
  \item At the end of the game, tell the user how long it took them to click all
    the circles. Recall that the method \texttt{System.currentTimeMillis()} is
    useful for finding how long something took.
  \item At the end of the game, show a dialog asking if the user would like to
    play again. If they say yes, start the game again from the beginning.
  \item In addition to displaying circles on the screen to be clicked, also add
    a number of squares which should \emph{not} be clicked. If the player clicks
    on a square, you could penalize their time or make them lose the game
    immediately. If you make this particular addition, two hints for you:
    \begin{itemize}
      \item You may find the keyword {\tt instanceof} useful. If {\tt o} is a
        GObject and you are not sure if it is a GRect or a GOval, you can write
        \verb|if (o instanceof GRect)|, where the condition holds only if {\tt
        o} is actually a GRect and not a GOval.
      \item Be careful not to draw squares which completely cover some of the
        circles, as this could make the game unwinnable. One way to avoid this
        is to draw the squares before drawing the circles, so that the circles
        will appear on top of the squares.
    \end{itemize}
  \item Instead of using random colors, use specific colors that mean something.
    Maybe you can keep score, and give more points for clicking several circles
    of the same color in a row.
  \item \textbf{Challenge} -- Make the circles move in some fashion, so that
    they are harder to click.
\end{itemize}
You will need to think carefully about what exactly needs to be stored in
instance variables. How will you detect when all circles have been clicked?
\end{document}
