\documentclass[11pt]{article}
\usepackage[noBBpl]{mathpazo}

%\documentclass[12pt]{article}
%\usepackage{mathptmx}

\usepackage{graphicx}
\usepackage{enumerate}
\usepackage{microtype}
\usepackage{amsfonts}
\usepackage{amsmath}
\usepackage{amssymb}
\usepackage{amsthm}
\usepackage{cancel}
\usepackage{mathrsfs}
\input xy
\xyoption{all}

\usepackage[noend]{algorithmic}
\usepackage{listings}
\lstset{language=C, tabsize=4, frame=single}

%\renewcommand{\labelenumi}{\textbf{(\arabic{enumi})}}

\pdfpagewidth 8.5in
\pdfpageheight 11in
\topmargin 0in
\headheight 0in
\headsep 0in
\textheight 9in
\oddsidemargin 0in
\evensidemargin 0in
\textwidth 6.5in

\theoremstyle{plain}
\newtheorem*{theorem}{Theorem}
\newtheorem*{lemma}{Lemma}
\newtheorem*{prop}{Proposition}
\newtheorem*{cor}{Corollary}

\theoremstyle{definition}
\newtheorem*{defn}{Definition}
\newtheorem*{prob}{Problem}
\newtheorem*{ex}{Example}
\newtheorem*{exes}{Examples}

\theoremstyle{remark}
\newtheorem*{remark}{Remark}
\newtheorem*{note}{Note}
\newtheorem*{claim}{Claim}
\newtheorem*{case}{Case}
\newtheorem*{conclusion}{Conclusion}

\newcommand{\N}{\mathbb N}
\newcommand{\Z}{\mathbb Z}
\newcommand{\Q}{\mathbb Q}
\newcommand{\R}{\mathbb R}
\newcommand{\C}{\mathbb C}

\newcommand{\paren}[1]{{\left({#1}\right)}}
\newcommand{\abs}[1]{{\left\lvert{#1}\right\rvert}}
\newcommand{\norm}[1]{{\left\lVert{#1}\right\rVert}}
\newcommand{\inner}[1]{{\left\langle{#1}\right\rangle}}
\newcommand{\floor}[1]{{\left\lfloor{#1}\right\rfloor}}
\newcommand{\ceil}[1]{{\left\lceil{#1}\right\rceil}}
\newcommand{\pd}[2]{{\frac{\partial{#1}}{\partial{#2}}}}
\newcommand{\unit}[1]{\,\mathrm{#1}}
\newcommand{\e}[1]{\times10^{#1}}
\newcommand{\bra}[1]{\langle #1 \rvert}
\newcommand{\ket}[1]{\lvert #1 \rangle}
\newcommand{\braket}[2]{\langle #1 | #2 \rangle}
\newcommand{\braaket}[3]{\bra{#1}#2\ket{#3}}

\DeclareMathOperator{\Real}{Re}
\DeclareMathOperator{\Imag}{Im}

\newcommand{\newproblem}[1]{\section*{\textsf{#1}\smallskip\hrule}}

\begin{document}
\begin{center}
\section*{Final Robotics Project}
\end{center}
It's time to make an awesome robot. With your group, make use of everything
you've learned to make a robot to perform some task. You are given significant
creative freedom as to what the robot should do, so think of something
interesting. The only requirement is that the robot must make use of both motors
and sensors. Other than that, you should try to come up with something cool.
Here are some ideas for what you can do:
\begin{itemize}
  \item Make a robot which has an arm! You can use the arm to pick up objects,
    or push them around.
  \item You can put a launcher on a robot to shoot balls forward. You can detect
    nearby targets and then shoot balls at them.
  \item If you have a light sensor facing straight down at the ground, and then
    put strips of black electrical tape on the ground, then the robot can detect
    when it is passing over tape. There are various things you can do with this.
    For example, you could use the strips of tape to make a maze on the ground,
    and then the robot can try to solve the maze without driving over any of the
    ``walls.''
  \item Using the ultrasonic sensor and possibly other sensors, make a robot
    which navigates around a room full of obstacles. You could make the
    robot try to move forward in a room as far as possible, moving around
    anything that gets in its path.
  \item Using the light sensor, make a robot which tries to move towards light.
    The robot will need to look around to determine which direction is
    brightest, and then move in that direction. You could then shine a
    flashlight to tell the robot where to go.
  \item The robot doesn't actually have to move around a room! Another option is
    a stationary robot that uses multiple motors to have precise control over an
    arm. You could make a robot which grips a dry-erase marker and uses it to
    write on a paper.
\end{itemize}
Your group will present your robot to the class on Friday, the last day of
class. Be excited!
\end{document}
