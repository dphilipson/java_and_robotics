\documentclass[11pt]{article}
\usepackage[noBBpl]{mathpazo}

%\documentclass[12pt]{article}
%\usepackage{mathptmx}

\usepackage{graphicx}
\usepackage{enumerate}
\usepackage{microtype}
\usepackage{amsfonts}
\usepackage{amsmath}
\usepackage{amssymb}
\usepackage{amsthm}
\usepackage{cancel}
\usepackage{mathrsfs}
\input xy
\xyoption{all}

\usepackage[noend]{algorithmic}
\usepackage{listings}
\lstset{language=C, tabsize=4, frame=single}

%\renewcommand{\labelenumi}{\textbf{(\arabic{enumi})}}

\pdfpagewidth 8.5in
\pdfpageheight 11in
\topmargin 0in
\headheight 0in
\headsep 0in
\textheight 9in
\oddsidemargin 0in
\evensidemargin 0in
\textwidth 6.5in

\theoremstyle{plain}
\newtheorem*{theorem}{Theorem}
\newtheorem*{lemma}{Lemma}
\newtheorem*{prop}{Proposition}
\newtheorem*{cor}{Corollary}

\theoremstyle{definition}
\newtheorem*{defn}{Definition}
\newtheorem*{prob}{Problem}
\newtheorem*{ex}{Example}
\newtheorem*{exes}{Examples}

\theoremstyle{remark}
\newtheorem*{remark}{Remark}
\newtheorem*{note}{Note}
\newtheorem*{claim}{Claim}
\newtheorem*{case}{Case}
\newtheorem*{conclusion}{Conclusion}

\newcommand{\N}{\mathbb N}
\newcommand{\Z}{\mathbb Z}
\newcommand{\Q}{\mathbb Q}
\newcommand{\R}{\mathbb R}
\newcommand{\C}{\mathbb C}

\newcommand{\paren}[1]{{\left({#1}\right)}}
\newcommand{\abs}[1]{{\left\lvert{#1}\right\rvert}}
\newcommand{\norm}[1]{{\left\lVert{#1}\right\rVert}}
\newcommand{\inner}[1]{{\left\langle{#1}\right\rangle}}
\newcommand{\floor}[1]{{\left\lfloor{#1}\right\rfloor}}
\newcommand{\ceil}[1]{{\left\lceil{#1}\right\rceil}}
\newcommand{\pd}[2]{{\frac{\partial{#1}}{\partial{#2}}}}
\newcommand{\unit}[1]{\,\mathrm{#1}}
\newcommand{\e}[1]{\times10^{#1}}
\newcommand{\bra}[1]{\langle #1 \rvert}
\newcommand{\ket}[1]{\lvert #1 \rangle}
\newcommand{\braket}[2]{\langle #1 | #2 \rangle}
\newcommand{\braaket}[3]{\bra{#1}#2\ket{#3}}

\DeclareMathOperator{\Real}{Re}
\DeclareMathOperator{\Imag}{Im}

\newcommand{\newproblem}[1]{\section*{\textsf{#1}\smallskip\hrule}}

\begin{document}
\begin{center}
\section*{Assignment 11 -- The Container classes}
\end{center}
\subsection*{Problem 1}
In {\tt ClosestNumber.java}, write a program which repeatedly prompts the user
for numbers until they enter -1. The program should then prompt the user for one
more number, and then tell them which of the initial numbers they entered was
closest. A sample run of the program might appear
\begin{quote}
  Enter a number (or -1 to continue): {\bf 2}\\
  Enter a number (or -1 to continue): {\bf 20}\\
  Enter a number (or -1 to continue): {\bf 28}\\
  Enter a number (or -1 to continue): {\bf 15}\\
  Enter a nubmer (or -1 to continue): {\bf -1}\\
  Enter one last number: {\bf 18}\\
  The number you entered closest to 18 was 20.
\end{quote}
If several numbers are tied for closest, you may output any of them.
\subsection*{Problem 2}
In {\tt AlphabeticalWords.java}, write a program which prints out all the words
in the file {\tt scrabble.txt} which are at least six letters long and whose
letters appear in alphabetical order.  To check your answer, the first few lines
of the output should appear
\begin{quote}
  abbess\\
  abhors\\
  accent\\
  accept
\end{quote}
Hints: recall that if {\tt s} is a String, then {\tt s.charAt(i)} returns the
character at position {\tt i}. Further, note that chars can be compared using
standard comparison operators, so for example the expression {\tt ('a' < 'c')}
evaluates to {\tt true}.
\subsection*{Project}
If you finish early, then continue to work on your final robotics project.
\end{document}
