\documentclass[11pt]{article}
\usepackage[noBBpl]{mathpazo}

%\documentclass[12pt]{article}
%\usepackage{mathptmx}

\usepackage{graphicx}
\usepackage{enumerate}
\usepackage{microtype}
\usepackage{amsfonts}
\usepackage{amsmath}
\usepackage{amssymb}
\usepackage{amsthm}
\usepackage{cancel}
\usepackage{mathrsfs}
\input xy
\xyoption{all}

\usepackage[noend]{algorithmic}
\usepackage{listings}
\lstset{language=C, tabsize=4, frame=single}

%\renewcommand{\labelenumi}{\textbf{(\arabic{enumi})}}

\pdfpagewidth 8.5in
\pdfpageheight 11in
\topmargin 0in
\headheight 0in
\headsep 0in
\textheight 9in
\oddsidemargin 0in
\evensidemargin 0in
\textwidth 6.5in

\theoremstyle{plain}
\newtheorem*{theorem}{Theorem}
\newtheorem*{lemma}{Lemma}
\newtheorem*{prop}{Proposition}
\newtheorem*{cor}{Corollary}

\theoremstyle{definition}
\newtheorem*{defn}{Definition}
\newtheorem*{prob}{Problem}
\newtheorem*{ex}{Example}
\newtheorem*{exes}{Examples}

\theoremstyle{remark}
\newtheorem*{remark}{Remark}
\newtheorem*{note}{Note}
\newtheorem*{claim}{Claim}
\newtheorem*{case}{Case}
\newtheorem*{conclusion}{Conclusion}

\newcommand{\N}{\mathbb N}
\newcommand{\Z}{\mathbb Z}
\newcommand{\Q}{\mathbb Q}
\newcommand{\R}{\mathbb R}
\newcommand{\C}{\mathbb C}

\newcommand{\paren}[1]{{\left({#1}\right)}}
\newcommand{\abs}[1]{{\left\lvert{#1}\right\rvert}}
\newcommand{\norm}[1]{{\left\lVert{#1}\right\rVert}}
\newcommand{\inner}[1]{{\left\langle{#1}\right\rangle}}
\newcommand{\floor}[1]{{\left\lfloor{#1}\right\rfloor}}
\newcommand{\ceil}[1]{{\left\lceil{#1}\right\rceil}}
\newcommand{\pd}[2]{{\frac{\partial{#1}}{\partial{#2}}}}
\newcommand{\unit}[1]{\,\mathrm{#1}}
\newcommand{\e}[1]{\times10^{#1}}
\newcommand{\bra}[1]{\langle #1 \rvert}
\newcommand{\ket}[1]{\lvert #1 \rangle}
\newcommand{\braket}[2]{\langle #1 | #2 \rangle}
\newcommand{\braaket}[3]{\bra{#1}#2\ket{#3}}

\DeclareMathOperator{\Real}{Re}
\DeclareMathOperator{\Imag}{Im}

\newcommand{\newproblem}[1]{\section*{\textsf{#1}\smallskip\hrule}}

\begin{document}
\begin{center}
\section*{Assignment 6 -- Basics of motors}
\end{center}
For this assignment, you should work in groups of three. Make sure that everyone
in your group has a chance to work at the computer.\\
\subsection*{Problem 1}
Write a program that starts the motor running forward and display the text
``Forward'' on the LCD screen. Each time a button is pressed, the direction of
the motor should reverse, and the LCD screen should change to say ``Forward'' or
``Backward'' depending on which direction the motor is currently moving.

Note that you can terminate a running program by pressing the enter and escape
buttons (the orange button and the one below it) at the same time.

\subsection*{Problem 2}
Build a cart-like robot which uses a single motor to turn wheels in order to
move forwards or backwards in a line. You will need to find a way to attach the
NXT brick to the robot so that it is carried with the robot as the robot moves.

\subsection*{Challenge problem}
Once you have a robot which can move itself, write a program which makes the
robot move forward as close to 5 feet as possible. If possible, use calculations
based on the radius of the wheels to determine how much the wheels should
rotate.

\end{document}
