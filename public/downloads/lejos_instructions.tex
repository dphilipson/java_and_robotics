\documentclass[11pt]{article}
\usepackage[noBBpl]{mathpazo}

%\documentclass[12pt]{article}
%\usepackage{mathptmx}

\usepackage{graphicx}
\usepackage{enumerate}
\usepackage{microtype}
\usepackage{amsfonts}
\usepackage{amsmath}
\usepackage{amssymb}
\usepackage{amsthm}
\usepackage{cancel}
\usepackage{mathrsfs}
\input xy
\xyoption{all}

\usepackage[noend]{algorithmic}
\usepackage{listings}
\lstset{language=Java, tabsize=4, frame=single}

%\renewcommand{\labelenumi}{\textbf{(\arabic{enumi})}}

\pdfpagewidth 8.5in
\pdfpageheight 11in
\topmargin 0in
\headheight 0in
\headsep 0in
\textheight 9in
\oddsidemargin 0in
\evensidemargin 0in
\textwidth 6.5in

\theoremstyle{plain}
\newtheorem*{theorem}{Theorem}
\newtheorem*{lemma}{Lemma}
\newtheorem*{prop}{Proposition}
\newtheorem*{cor}{Corollary}

\theoremstyle{definition}
\newtheorem*{defn}{Definition}
\newtheorem*{prob}{Problem}
\newtheorem*{ex}{Example}
\newtheorem*{exes}{Examples}

\theoremstyle{remark}
\newtheorem*{remark}{Remark}
\newtheorem*{note}{Note}
\newtheorem*{claim}{Claim}
\newtheorem*{case}{Case}
\newtheorem*{conclusion}{Conclusion}

\newcommand{\N}{\mathbb N}
\newcommand{\Z}{\mathbb Z}
\newcommand{\Q}{\mathbb Q}
\newcommand{\R}{\mathbb R}
\newcommand{\C}{\mathbb C}

\newcommand{\paren}[1]{{\left({#1}\right)}}
\newcommand{\abs}[1]{{\left\lvert{#1}\right\rvert}}
\newcommand{\norm}[1]{{\left\lVert{#1}\right\rVert}}
\newcommand{\inner}[1]{{\left\langle{#1}\right\rangle}}
\newcommand{\floor}[1]{{\left\lfloor{#1}\right\rfloor}}
\newcommand{\ceil}[1]{{\left\lceil{#1}\right\rceil}}
\newcommand{\pd}[2]{{\frac{\partial{#1}}{\partial{#2}}}}
\newcommand{\unit}[1]{\,\mathrm{#1}}
\newcommand{\e}[1]{\times10^{#1}}
\newcommand{\bra}[1]{\langle #1 \rvert}
\newcommand{\ket}[1]{\lvert #1 \rangle}
\newcommand{\braket}[2]{\langle #1 | #2 \rangle}
\newcommand{\braaket}[3]{\bra{#1}#2\ket{#3}}

\DeclareMathOperator{\Real}{Re}
\DeclareMathOperator{\Imag}{Im}

\newcommand{\newproblem}[1]{\section*{\textsf{#1}\smallskip\hrule}}

\begin{document}
\begin{center}
\section*{How to use LeJOS}
\end{center}
Unfortunately, the Eclipse plugin we have been using up until now wasn't
designed to work with LeJOS, so creating and running projects will work a little
differently.\\
\subsection*{Creating a LeJOS project}
\begin{itemize}
  \item In Eclipse, on the menu at the top of the window select File $\to$ New
    $\to$ Project.
  \item On the menu that appears, select LeJOS $\to$ LeJOS NXT Project. Click
    Next.
  \item On the next screen, enter any name you want, then click Finish.
  \item Select the project you just created in the left-hand pane (it should
    have a small LeJOS logo over the folder icon, to indicate that it is a LeJOS
    project).
  \item In the File menu at the top of the screen, or the one that appears by
    right-clicking the project, select New $\to$ Class.
  \item Enter a name for the class, describing the type of program you're
    writing.
  \item The class appears in your editor. Inside the class, create a method
    which appears
    \begin{lstlisting}
public static void main(String[] args) {
    // Your code goes here.
}\end{lstlisting}
\end{itemize}
You're ready to write your program! If you want to create more programs in the
same project, you can create more classes by following the last three steps
again.
\subsection*{Running a LeJOS program}
To run a LeJOS program that you have just written, follow these steps:
\begin{itemize}
  \item Connect your NXT brick to your computer via a USB cable and turn on the
    brick by pressing the Enter button (the big orange one).
  \item Right-click your program, and on the menu that appears select Run As
    $\to$ LeJOS NXT Program.
\end{itemize}
Depending on the settings in your editor, this will upload your program and
immediately run it, or simply upload your program. In any case, once the program
is uploaded to your brick, you can run it by selecting Files on the NXT menu,
then selecting the program you want to run, then selecting Execute Program.
\end{document}
