\documentclass[11pt]{article}
\usepackage[noBBpl]{mathpazo}

%\documentclass[12pt]{article}
%\usepackage{mathptmx}

\usepackage{enumerate}
\usepackage{microtype}
\usepackage{amsfonts}
\usepackage{amsmath}
\usepackage{amssymb}
\usepackage{amsthm}
\usepackage{cancel}
\usepackage{mathrsfs}
\input xy
\xyoption{all}

\usepackage[noend]{algorithmic}
\usepackage{listings}
\lstset{language=C, tabsize=4, frame=single}

%\renewcommand{\labelenumi}{\textbf{(\arabic{enumi})}}

\pdfpagewidth 8.5in
\pdfpageheight 11in
\topmargin 0in
\headheight 0in
\headsep 0in
\textheight 9in
\oddsidemargin 0in
\evensidemargin 0in
\textwidth 6.5in

\theoremstyle{plain}
\newtheorem*{theorem}{Theorem}
\newtheorem*{lemma}{Lemma}
\newtheorem*{prop}{Proposition}
\newtheorem*{cor}{Corollary}

\theoremstyle{definition}
\newtheorem*{defn}{Definition}
\newtheorem*{prob}{Problem}
\newtheorem*{ex}{Example}
\newtheorem*{exes}{Examples}

\theoremstyle{remark}
\newtheorem*{remark}{Remark}
\newtheorem*{note}{Note}
\newtheorem*{claim}{Claim}
\newtheorem*{case}{Case}
\newtheorem*{conclusion}{Conclusion}

\newcommand{\N}{\mathbb N}
\newcommand{\Z}{\mathbb Z}
\newcommand{\Q}{\mathbb Q}
\newcommand{\R}{\mathbb R}
\newcommand{\C}{\mathbb C}

\newcommand{\paren}[1]{{\left({#1}\right)}}
\newcommand{\abs}[1]{{\left\lvert{#1}\right\rvert}}
\newcommand{\norm}[1]{{\left\lVert{#1}\right\rVert}}
\newcommand{\inner}[1]{{\left\langle{#1}\right\rangle}}
\newcommand{\floor}[1]{{\left\lfloor{#1}\right\rfloor}}
\newcommand{\ceil}[1]{{\left\lceil{#1}\right\rceil}}
\newcommand{\pd}[2]{{\frac{\partial{#1}}{\partial{#2}}}}
\newcommand{\unit}[1]{\,\mathrm{#1}}
\newcommand{\e}[1]{\times10^{#1}}
\newcommand{\bra}[1]{\langle #1 \rvert}
\newcommand{\ket}[1]{\lvert #1 \rangle}
\newcommand{\braket}[2]{\langle #1 | #2 \rangle}
\newcommand{\braaket}[3]{\bra{#1}#2\ket{#3}}

\DeclareMathOperator{\Real}{Re}
\DeclareMathOperator{\Imag}{Im}

\newcommand{\newproblem}[1]{\section*{\textsf{#1}\smallskip\hrule}}

\begin{document}
\begin{center}
\section*{Assignment 2 -- Working with Variables}
\end{center}
\subsection*{Problem 1 -- A simple greeting}
This problem will give you a chance to get your hands dirty with a
ConsoleProgram. Write a program which asks the user for his or her name. The
program should reply with a greeting to that user. In addition, if the user has
your own name, you should reply with a special greeting. A sample run of the
program might look like this, where the bold text has been entered by the user.
\begin{quote}
  Hello! What is your name? \textbf{Thor}\\
  Hi Thor, it's good to meet you.
\end{quote}
Another run might appear
\begin{quote}
  Hello! What is your name? \textbf{David}\\
  Weird, that's my name too.
\end{quote}
You do not need to copy the wording exactly. Feel free to write your own
greetings and responses, as long as it meets the requirements in the above
paragraph.
\subsection*{Problem 2 -- Computing variance}
In lecture, we saw how to write a program which computes the average of a list
of numbers input by the user. In this problem, we will modify that program to
instead compute the variance.

The \textbf{variance} of a list of numbers is a measure of how far those numbers
are from the mean, or average. A high variance indicates that the numbers are
widely spread out, while a low variance indicates that the numbers are all very
close to each other. In particular, a variance of $0$ occurs exactly when all of
the numbers are exactly the same, The variance is never negative.

For a list of numbers $x_1, x_2, \ldots, x_n$, the variance may be
computed by the formula
\[
\frac{x_1^2 + x_2^2 + \cdots + x_n^2}{n} - \paren{\frac{x_1 + x_2 + \cdots
+ x_n}{n}}^2
\]
For example, the variance of the numbers $10, 11, 12$ is computed as
\[
\frac{10^2 + 11^2 + 12^2}{3} - \paren{\frac{10 + 11 + 12}{3}}^2
\]
which turns out to be $2/3$.

Another way you may find helpful to think of this computation: the first term is
the average of the squares of the inputs. The second term is the square of the
average of the inputs.

You should write a program which accepts numeric inputs from the user until the
user enters $-1$, at which point you should print out the variance of all the
numbers entered before the $-1$. A sample run might appear as follows:
\begin{quote}
  Enter a number (or enter -1 to finish): \textbf{4} \\
  Enter a number (or enter -1 to finish): \textbf{4} \\
  Enter a number (or enter -1 to finish): \textbf{5} \\
  Enter a number (or enter -1 to finish): \textbf{6} \\
  Enter a number (or enter -1 to finish): \textbf{-1} \\
  The variance is 0.6875.
\end{quote}
As a side note, the square root of the variance is known as \textbf{standard
deviation}, and is commonly used when discussing statistics (such as test
scores) to indicate how widely spread the data is.
\subsection*{Challenge Problem -- Reverse guessing game}
In class, we wrote a program which challenged the user to try to guess its
hidden number, and at each step the program told the user whether the guess was
high or low. In this problem, we write a program to play the same game, but from
the other side. The user thinks of a random number, and the program makes
guesses. After each guess, the user tells the program whether the guess was
high, low, or correct. A sample run might appear as follows:
\begin{quote}
  Think of a number between 1 and 100.
  I guess 50. Is that high, low, or correct? \textbf{high} \\
  I guess 25. Is that high, low, or correct? \textbf{high} \\
  I guess 12. Is that high, low, or correct? \textbf{low} \\
  I guess 18. Is that high, low, or correct? \textbf{high} \\
  I guess 15. Is that high, low, or correct? \textbf{correct} \\
  Good game!
\end{quote}
The program's strategy is to keep track of a range in which it knows the number
is in, and to always guess at the middle of that range. For example, in the
above run it followed the following reasoning:
\begin{itemize}
  \item The initial range is 1 to 100. The middle of this range is $(1 + 100) /
    2$, which I'll round down to $50$.
  \item $50$ was too high. That means I now know the number is in the range 1 to
    49. The middle of this range is $(1 + 49) / 2$, so I'll guess $25$.
  \item $25$ was too high. That means I now know the number is in the range 1 to
    24. The middle of this range is $(1 + 24) / 2$, which I'll round down to
    $12$.
  \item $12$ was too low. This means I now know the number is in the range 13 to
    24. The middle of this range is $(13 + 24) / 2$, which I'll round down to
    $18$.
  \item $18$ was too high. So the number must be in the range 13 to 18. The
    middle of this range is $(13 + 18) / 2$, which I'll round down to $15$.
\end{itemize}
It's a very good  accomplishment to get a basic version of the program running.
If you can do that, you can try adding the following features:
\begin{itemize}
  \item If the user enters something other than ``high'', ``low'', or
    ``correct'', prompt them again until they enter one of the acceptable
    values.
  \item If the user has given input which cannot describe any actual number (for
    example, the user never says that any guess is correct, leading to a range
    from 25 to 24), then the program should let the user know that they are a
    dirty, dirty cheater.
\end{itemize}
\end{document}
