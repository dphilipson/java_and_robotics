\documentclass[11pt]{article}
\usepackage[noBBpl]{mathpazo}

%\documentclass[12pt]{article}
%\usepackage{mathptmx}

\usepackage{graphicx}
\usepackage{enumerate}
\usepackage{microtype}
\usepackage{amsfonts}
\usepackage{amsmath}
\usepackage{amssymb}
\usepackage{amsthm}
\usepackage{cancel}
\usepackage{mathrsfs}
\input xy
\xyoption{all}

\usepackage[noend]{algorithmic}
\usepackage{listings}
\lstset{language=C, tabsize=4, frame=single}

%\renewcommand{\labelenumi}{\textbf{(\arabic{enumi})}}

\pdfpagewidth 8.5in
\pdfpageheight 11in
\topmargin 0in
\headheight 0in
\headsep 0in
\textheight 9in
\oddsidemargin 0in
\evensidemargin 0in
\textwidth 6.5in

\theoremstyle{plain}
\newtheorem*{theorem}{Theorem}
\newtheorem*{lemma}{Lemma}
\newtheorem*{prop}{Proposition}
\newtheorem*{cor}{Corollary}

\theoremstyle{definition}
\newtheorem*{defn}{Definition}
\newtheorem*{prob}{Problem}
\newtheorem*{ex}{Example}
\newtheorem*{exes}{Examples}

\theoremstyle{remark}
\newtheorem*{remark}{Remark}
\newtheorem*{note}{Note}
\newtheorem*{claim}{Claim}
\newtheorem*{case}{Case}
\newtheorem*{conclusion}{Conclusion}

\newcommand{\N}{\mathbb N}
\newcommand{\Z}{\mathbb Z}
\newcommand{\Q}{\mathbb Q}
\newcommand{\R}{\mathbb R}
\newcommand{\C}{\mathbb C}

\newcommand{\paren}[1]{{\left({#1}\right)}}
\newcommand{\abs}[1]{{\left\lvert{#1}\right\rvert}}
\newcommand{\norm}[1]{{\left\lVert{#1}\right\rVert}}
\newcommand{\inner}[1]{{\left\langle{#1}\right\rangle}}
\newcommand{\floor}[1]{{\left\lfloor{#1}\right\rfloor}}
\newcommand{\ceil}[1]{{\left\lceil{#1}\right\rceil}}
\newcommand{\pd}[2]{{\frac{\partial{#1}}{\partial{#2}}}}
\newcommand{\unit}[1]{\,\mathrm{#1}}
\newcommand{\e}[1]{\times10^{#1}}
\newcommand{\bra}[1]{\langle #1 \rvert}
\newcommand{\ket}[1]{\lvert #1 \rangle}
\newcommand{\braket}[2]{\langle #1 | #2 \rangle}
\newcommand{\braaket}[3]{\bra{#1}#2\ket{#3}}

\DeclareMathOperator{\Real}{Re}
\DeclareMathOperator{\Imag}{Im}

\newcommand{\newproblem}[1]{\section*{\textsf{#1}\smallskip\hrule}}

\begin{document}
\begin{center}
\section*{Assignment 9 -- Creating classes}
\end{center}
\subsubsection*{Problem 1}
In the file {\tt Circle.java}, fill in the code to define a class {\tt Circle}
which represents a circle of a given radius and has methods which provide
information about its current radius, area, and circumference. In particular, it
has the following constructor and methods:
\begin{itemize}
  \item Its constructor takes a single double, representing the radius.
  \item A method {\tt getRadius()}, which takes no arguments and returns the
    radius as a double.
  \item A method {\tt getCircumference()}, which takes no arguments and returns
    the circumference as a double (the circumference of a circle is $2\pi r$,
    where $r$ is the radius).
  \item A method {\tt getArea()}, which takes no arguments and returns the area
    as a double (the area of a circle is $\pi r^2$, where $r$ is the radius).
  \item A method {\tt equals()}, which takes another {\tt Circle} as its
    argument and returns true if the other circle has the same radius.
\end{itemize}
The class should have no public instance variables. In particular, there is no
way to change the radius of the circle after the circle has been
created.\footnote{Because of this, we say that {\tt Circle} is
\emph{immutable}.}

The file {\tt TestCircle.java} contains a ConsoleProgram which can be used to
test your circle implementation to see if it is working correctly. You can feel
free to examine or modify this code.
\subsubsection*{Problem 2}
In lecture, we saw how to create a {\tt MovingBall} class. Along similar lines,
you should now fill in code in {\tt BouncingBall.java} to create a new class
called {\tt BouncingBall} which represents a GOval bouncing around the screen.
You must implement the following constructor and method:
\begin{itemize}
  \item A constructor which takes three arguments: a GOval which is to bounce
    around the screen, and two doubles representing the initial x- and
    y-velocity.
  \item A method {\tt move()}, which takes no arguments. When this method is
    called, the GOval should be moved on the screen according to the current
    velocity. If this would cause the GOval to contact any side of the window,
    then the GOval should ``bounce'' off that side, reversing the x-velocity if
    on the left- or right-edge, and reversing the y-velocity if on the top- or
    bottom-edge.
\end{itemize}
Hence, when the {\tt move()} method is called repeatedly, the given GOval
should appear to travel around the window, bouncing off sides when it makes
contact.

Once you have defined your class, you should write a GraphicsProgram in {\tt
BouncingBallsProgram.java} which displays several balls of different colors
bouncing around the screen. You may want to randomize their start locations
and/or their initial velocities.
\subsubsection*{Challenge problem}
Extend the bouncing balls program from the previous part in the following ways:
\begin{itemize}
  \item Add an additional method to {\tt BouncingBall} class called {\tt
    isContacting()}, which takes in another {\tt BouncingBall} as an argument
    and returns a boolean which is true if the two circles are overlapping, and
    false if not.
  \item Using this new method, modify your {\tt BouncingBallsProgram} so that
    something happens if two balls run into each other. This could be as simple
    as changing their colors, or you could have the balls bounce off of each
    other in opposite directions.
\end{itemize}
A hint: two circles are overlapping if the distance between their centers is
less than the sums of their radii. The distance between two points $(x_1, y_1)$
and $(x_2, y_2)$, according to the Pythagorean theorem, is $\sqrt{(x_2 - x_1)^2
+ (y_2 - y_1)^2}$.

\end{document}
