\documentclass[11pt]{article}
\usepackage[noBBpl]{mathpazo}

%\documentclass[12pt]{article}
%\usepackage{mathptmx}

\usepackage{graphicx}
\usepackage{enumerate}
\usepackage{microtype}
\usepackage{amsfonts}
\usepackage{amsmath}
\usepackage{amssymb}
\usepackage{amsthm}
\usepackage{cancel}
\usepackage{mathrsfs}
\input xy
\xyoption{all}

\usepackage[noend]{algorithmic}
\usepackage{listings}
\lstset{language=C, tabsize=4, frame=single}

%\renewcommand{\labelenumi}{\textbf{(\arabic{enumi})}}

\pdfpagewidth 8.5in
\pdfpageheight 11in
\topmargin 0in
\headheight 0in
\headsep 0in
\textheight 9in
\oddsidemargin 0in
\evensidemargin 0in
\textwidth 6.5in

\theoremstyle{plain}
\newtheorem*{theorem}{Theorem}
\newtheorem*{lemma}{Lemma}
\newtheorem*{prop}{Proposition}
\newtheorem*{cor}{Corollary}

\theoremstyle{definition}
\newtheorem*{defn}{Definition}
\newtheorem*{prob}{Problem}
\newtheorem*{ex}{Example}
\newtheorem*{exes}{Examples}

\theoremstyle{remark}
\newtheorem*{remark}{Remark}
\newtheorem*{note}{Note}
\newtheorem*{claim}{Claim}
\newtheorem*{case}{Case}
\newtheorem*{conclusion}{Conclusion}

\newcommand{\N}{\mathbb N}
\newcommand{\Z}{\mathbb Z}
\newcommand{\Q}{\mathbb Q}
\newcommand{\R}{\mathbb R}
\newcommand{\C}{\mathbb C}

\newcommand{\paren}[1]{{\left({#1}\right)}}
\newcommand{\abs}[1]{{\left\lvert{#1}\right\rvert}}
\newcommand{\norm}[1]{{\left\lVert{#1}\right\rVert}}
\newcommand{\inner}[1]{{\left\langle{#1}\right\rangle}}
\newcommand{\floor}[1]{{\left\lfloor{#1}\right\rfloor}}
\newcommand{\ceil}[1]{{\left\lceil{#1}\right\rceil}}
\newcommand{\pd}[2]{{\frac{\partial{#1}}{\partial{#2}}}}
\newcommand{\unit}[1]{\,\mathrm{#1}}
\newcommand{\e}[1]{\times10^{#1}}
\newcommand{\bra}[1]{\langle #1 \rvert}
\newcommand{\ket}[1]{\lvert #1 \rangle}
\newcommand{\braket}[2]{\langle #1 | #2 \rangle}
\newcommand{\braaket}[3]{\bra{#1}#2\ket{#3}}

\DeclareMathOperator{\Real}{Re}
\DeclareMathOperator{\Imag}{Im}

\newcommand{\newproblem}[1]{\section*{\textsf{#1}\smallskip\hrule}}

\begin{document}
\begin{center}
\section*{Assignment 8 -- Sensors}
\end{center}
\subsection*{Problem 1}
Write a program for your robot which uses a touch sensor (or multiple touch
sensors) to correct its path when encountering obstacles. In particular, your
robot should do the following:
\begin{itemize}
  \item Begin by traveling forward.
  \item The touch sensor(s) should be placed so that it is pressed when the
    robot runs into a wall or other obstacle. When this happens, the robot
    should back up a short distance, then turn ninety degrees to the left.
  \item The robot should now start traveling forward again until it hits another
    wall, and so on indefinitely.
\end{itemize}
Don't forget that you can always kill a running program by pressing the enter
and escape buttons (the orange button and the one below it) at the same time.
\subsection*{Problem 2}
Write a program for your robot that uses a sound sensor to react to loud noises.
Your robot should do the following:
\begin{itemize}
  \item Begin by traveling forward.
  \item Whenever the robot hears a loud sound, such as you shouting at it, it
    should stop moving, then turn ninety degrees to the right, then start
    moving again.
\end{itemize}
Part of this problem is figuring out what exactly should count as a ``loud
noise.'' Ideally, the robot should be fairly sensitive to sounds, but not so
sensitive that the noise of its own motors or ambient sounds causes it to make
durns.
\subsection*{Challenge problem}
Combine the two behaviors from the previous parts into a single program. That
is, the robot should move forward indefinitely, but whenever it encounters a
wall it should turn left, and whenever it hears a loud noise it turns right.

Now design an obstacle course by putting up barriers and choosing a start and
finish location. To complete the course, you must navigate the robot through the
course by making noises as necessary to tell it to turn.

Let other groups try out your obstacle course by controlling your robot with
sound. You can have other groups compete to finish your course in the shortest
time.

\end{document}
