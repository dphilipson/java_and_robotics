\documentclass[11pt]{article}
\usepackage[noBBpl]{mathpazo}

%\documentclass[12pt]{article}
%\usepackage{mathptmx}

\usepackage{graphicx}
\usepackage{enumerate}
\usepackage{microtype}
\usepackage{amsfonts}
\usepackage{amsmath}
\usepackage{amssymb}
\usepackage{amsthm}
\usepackage{cancel}
\usepackage{mathrsfs}
\input xy
\xyoption{all}

\usepackage[noend]{algorithmic}
\usepackage{listings}
\lstset{language=C, tabsize=4, frame=single}

%\renewcommand{\labelenumi}{\textbf{(\arabic{enumi})}}

\pdfpagewidth 8.5in
\pdfpageheight 11in
\topmargin 0in
\headheight 0in
\headsep 0in
\textheight 9in
\oddsidemargin 0in
\evensidemargin 0in
\textwidth 6.5in

\theoremstyle{plain}
\newtheorem*{theorem}{Theorem}
\newtheorem*{lemma}{Lemma}
\newtheorem*{prop}{Proposition}
\newtheorem*{cor}{Corollary}

\theoremstyle{definition}
\newtheorem*{defn}{Definition}
\newtheorem*{prob}{Problem}
\newtheorem*{ex}{Example}
\newtheorem*{exes}{Examples}

\theoremstyle{remark}
\newtheorem*{remark}{Remark}
\newtheorem*{note}{Note}
\newtheorem*{claim}{Claim}
\newtheorem*{case}{Case}
\newtheorem*{conclusion}{Conclusion}

\newcommand{\N}{\mathbb N}
\newcommand{\Z}{\mathbb Z}
\newcommand{\Q}{\mathbb Q}
\newcommand{\R}{\mathbb R}
\newcommand{\C}{\mathbb C}

\newcommand{\paren}[1]{{\left({#1}\right)}}
\newcommand{\abs}[1]{{\left\lvert{#1}\right\rvert}}
\newcommand{\norm}[1]{{\left\lVert{#1}\right\rVert}}
\newcommand{\inner}[1]{{\left\langle{#1}\right\rangle}}
\newcommand{\floor}[1]{{\left\lfloor{#1}\right\rfloor}}
\newcommand{\ceil}[1]{{\left\lceil{#1}\right\rceil}}
\newcommand{\pd}[2]{{\frac{\partial{#1}}{\partial{#2}}}}
\newcommand{\unit}[1]{\,\mathrm{#1}}
\newcommand{\e}[1]{\times10^{#1}}
\newcommand{\bra}[1]{\langle #1 \rvert}
\newcommand{\ket}[1]{\lvert #1 \rangle}
\newcommand{\braket}[2]{\langle #1 | #2 \rangle}
\newcommand{\braaket}[3]{\bra{#1}#2\ket{#3}}

\DeclareMathOperator{\Real}{Re}
\DeclareMathOperator{\Imag}{Im}

\newcommand{\newproblem}[1]{\section*{\textsf{#1}\smallskip\hrule}}

\begin{document}
\begin{center}
\section*{Assignment 5 -- Free project}
\end{center}
Congratulations to everyone for making it to the end of the first week! We've
covered a lot, and now's your chance to have some fun and be creative.

In this assignment, use any of the things we have learned so far in this class
to make a program of your choice. Some possibilities include:
\begin{itemize}
  \item A ``choose your own adventure'' or fortune-telling game in a
    ConsoleProgram. You can tell a story and use the player's choices to say
    what happens, or you can ask the player questions and then make predictions
    about their future.
  \item A crazy animation. You could have some kind of animation involving many
    bouncing balls, or shapes which grow and shrink, or words which appear on
    the screen.
  \item Some kind of mouse-based game. You could have a clicking game similar to
    RandomCircles from the last assignment, or you could have the player have to
    move the mouse from one side to the other while trying to avoid touching any
    boxes on the screen, or some kind of reflex-based game where an object
    appears and the player tries to click it as fast as possible.
  \item Some kind of keyboard-based game. You could allow the player to move a
    box around the screen using the W, A, S, D keys on the keyboard, or have
    letters appear on the screen and have the player have to type them as fast
    as possible.
  \item An awesome still image. Draw a picture using GObjects. If you want to
    practice your recursion, many incredibly complex pictures called
    \emph{fractals} can be made with only a few lines of code.
\end{itemize}

\end{document}
